% mnras_template.tex
%
% LaTeX template for creating an MNRAS paper
%
% v3.0 released 14 May 2015
% (version numbers match those of mnras.cls)
%
% Copyright (C) Royal Astronomical Society 2015
% Authors:
% Keith T. Smith (Royal Astronomical Society)

% Change log
%
% v3.0 May 2015
%    Renamed to match the new package name
%    Version number matches mnras.cls
%    A few minor tweaks to wording
% v1.0 September 2013
%    Beta testing only - never publicly released
%    First version: a simple (ish) template for creating an MNRAS paper

%%%%%%%%%%%%%%%%%%%%%%%%%%%%%%%%%%%%%%%%%%%%%%%%%%
% Basic setup. Most papers should leave these options alone.
\documentclass[fleqn,usenatbib]{mnras}

% MNRAS is set in Times font. If you don't have this installed (most LaTeX
% installations will be fine) or prefer the old Computer Modern fonts, comment
% out the following line
\usepackage{newtxtext,newtxmath}
% Depending on your LaTeX fonts installation, you might get better results with one of these:
%\usepackage{mathptmx}
%\usepackage{txfonts}

% Use vector fonts, so it zooms properly in on-screen viewing software
% Don't change these lines unless you know what you are doing
\usepackage[T1]{fontenc}
\usepackage{ae,aecompl}


%%%%% AUTHORS - PLACE YOUR OWN PACKAGES HERE %%%%%
\usepackage{bm}
\usepackage[usenames, dvipsnames]{color}
\usepackage{subcaption}
\captionsetup{compatibility=false}
% Only include extra packages if you really need them. Common packages are:
\usepackage{graphicx}	% Including figure files
\usepackage{amsmath}	% Advanced maths commands
\usepackage{amssymb}	% Extra maths symbols

%%%%%%%%%%%%%%%%%%%%%%%%%%%%%%%%%%%%%%%%%%%%%%%%%%

%%%%% AUTHORS - PLACE YOUR OWN COMMANDS HERE %%%%%
\graphicspath{{./Plots/}}
% Please keep new commands to a minimum, and use \newcommand not \def to avoid
% overwriting existing commands. Example:
%\newcommand{\pcm}{\,cm$^{-2}$}	% per cm-squared

%%%%%%%%%%%%%%%%%%%%%%%%%%%%%%%%%%%%%%%%%%%%%%%%%%

%%%%%%%%%%%%%%%%%%% TITLE PAGE %%%%%%%%%%%%%%%%%%%

% Title of the paper, and the short title which is used in the headers.
% Keep the title short and informative.
\title[]{Report Jan 2020 – Attempts at parameterising an empirical AUF}

% The list of authors, and the short list which is used in the headers.
% If you need two or more lines of authors, add an extra line using \newauthor
\author[Tom J. Wilson and Tim Naylor]{
Tom J. Wilson
and Tim Naylor
\\
}

% These dates will be filled out by the publisher
\date{}

% Enter the current year, for the copyright statements etc.
\pubyear{2020}

% Don't change these lines
\begin{document}
\label{firstpage}
\pagerange{\pageref{firstpage}--\pageref{lastpage}}
\maketitle
%%%%%%%%%%%%%%%%%%%%%%%%%%%%%%%%%%%%%%%%%%%%%%%%%%%%%%%%%%%%%%%%%%%%%%%%%%%%%%%%
\begin{abstract}

Given the depths that LSST will reach, we possibly require a different approach than previously undertaken to compute the perturbation component of the Astrometric Uncertainty Function (AUF) for probabilistic photometric catalogue cross-matching. We document here an attempt to derive this AUF -- along with any other non-statistical AUF component, such as proper motions or binary reflex motion -- from the data themselves, creating in-situ AUFs which do not rely on simulation.

\end{abstract}


%%%%%%%%%%%%%%%%%%%%%%%%%%%%%%%%%%%%%%%%%%%%%%%%%%%%%%%%%%%%%%%%%%%%%%%%%%%%%%%%
\section{Introduction}

The Astrometric Uncertainty Function (AUF) is the probability density function (PDF) describing the probability of a source being detected at some position given its true location. The probability of two sources being the same intrinsic object detected twice given their respective locations -- or the distance between their given positions -- is the convolution of their respective AUFs. Each AUF is, itself, a convolution of the individual components describing its offset from its ``true'' location. The first of these, and typically the only component used, is its statistical component, in which Poisson noise in a photometric image causes the centroiding of a source imperfectly.

A second critical component in crowded regions, or in photometric surveys which probe significant depths relative to the length scale of the survey PSF (of which LSST and \textit{WISE} are both subject), is the perturbation from blended sources. These objects, while hidden beneath the flux from a brighter source, contaminant the central object by both adding additional flux, and by tugging on the center-of-light of the composite object. Thus, in this simple two-component model, the AUF is

\begin{equation}
h(\textbf{x}) = (f_\mathrm{stat} * f_\mathrm{blend})(\textbf{x})
\end{equation}
where the notation $(f * g)(x)$ denotes a convolution.

Previously, \citet{2018MNRAS.481.2148W} produced $f_\mathrm{blend}$ by a Monte Carlo simulation, via Galactic stellar densities from a TRILEGAL simulation \citep{Girardi2005}. However, the limiting magnitude of these simulations -- via the public API -- is 32nd magnitude; this was fine for \textit{WISE} where we still had access to stars 10 magnitudes fainter than the limit of the survey (necessary for probing the faintest flux ratios and sufficiently small flux weighted astrometric perturbations), being brighter than $W1=32$. However, the limiting magnitude of LSST is about 27th magnitude, and thus the magnitude limit is out of the range of the TRILEGAL simulations.

\section{In situ AUF}
Thus the need for deriving empirical AUFs was born. 

%%%%%%%%%%%%%%%%%%%%%%%%%%%%%%%%%%%%%%%%%%%%%%%%%%%%%%%%%%%%%%%%%%%%%%%%%%%%%%%%

\bibliographystyle{mnras}
\bibliography{../../../../Dropbox/PostDoc_Exeter/LaTeX/postdoc_papers_jr2}



% Don't change these lines
\bsp	% typesetting comment
\label{lastpage}
\end{document}

% End of mnras_template.tex