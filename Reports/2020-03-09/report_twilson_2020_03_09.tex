% mnras_template.tex
%
% LaTeX template for creating an MNRAS paper
%
% v3.0 released 14 May 2015
% (version numbers match those of mnras.cls)
%
% Copyright (C) Royal Astronomical Society 2015
% Authors:
% Keith T. Smith (Royal Astronomical Society)

% Change log
%
% v3.0 May 2015
%    Renamed to match the new package name
%    Version number matches mnras.cls
%    A few minor tweaks to wording
% v1.0 September 2013
%    Beta testing only - never publicly released
%    First version: a simple (ish) template for creating an MNRAS paper

%%%%%%%%%%%%%%%%%%%%%%%%%%%%%%%%%%%%%%%%%%%%%%%%%%
% Basic setup. Most papers should leave these options alone.
\documentclass[fleqn,usenatbib]{mnras}

% MNRAS is set in Times font. If you don't have this installed (most LaTeX
% installations will be fine) or prefer the old Computer Modern fonts, comment
% out the following line
\usepackage{newtxtext,newtxmath}
% Depending on your LaTeX fonts installation, you might get better results with one of these:
%\usepackage{mathptmx}
%\usepackage{txfonts}

% Use vector fonts, so it zooms properly in on-screen viewing software
% Don't change these lines unless you know what you are doing
\usepackage[T1]{fontenc}
\usepackage{ae,aecompl}


%%%%% AUTHORS - PLACE YOUR OWN PACKAGES HERE %%%%%
\usepackage{bm}
\usepackage[usenames, dvipsnames]{color}
\usepackage{subcaption}
\captionsetup{compatibility=false}
% Only include extra packages if you really need them. Common packages are:
\usepackage{graphicx}	% Including figure files
\usepackage{amsmath}	% Advanced maths commands
\usepackage{amssymb}	% Extra maths symbols

%%%%%%%%%%%%%%%%%%%%%%%%%%%%%%%%%%%%%%%%%%%%%%%%%%

%%%%% AUTHORS - PLACE YOUR OWN COMMANDS HERE %%%%%
\graphicspath{{./Plots/}}
% Please keep new commands to a minimum, and use \newcommand not \def to avoid
% overwriting existing commands. Example:
%\newcommand{\pcm}{\,cm$^{-2}$}	% per cm-squared

%%%%%%%%%%%%%%%%%%%%%%%%%%%%%%%%%%%%%%%%%%%%%%%%%%

%%%%%%%%%%%%%%%%%%% TITLE PAGE %%%%%%%%%%%%%%%%%%%

% Title of the paper, and the short title which is used in the headers.
% Keep the title short and informative.
\title[]{Report Mar 2020 – Improving the perturbation AUF for faint sources}

% The list of authors, and the short list which is used in the headers.
% If you need two or more lines of authors, add an extra line using \newauthor
\author[Tom J. Wilson and Tim Naylor]{
Tom J. Wilson
and Tim Naylor
\\
}

% These dates will be filled out by the publisher
\date{}

% Enter the current year, for the copyright statements etc.
\pubyear{2020}

% Don't change these lines
\begin{document}
\label{firstpage}
\pagerange{\pageref{firstpage}--\pageref{lastpage}}
\maketitle
%%%%%%%%%%%%%%%%%%%%%%%%%%%%%%%%%%%%%%%%%%%%%%%%%%%%%%%%%%%%%%%%%%%%%%%%%%%%%%%%
\begin{abstract}

While we were unable to create physically motivated empirical AUFs from the data in situ, this may not necessarily be required. The current model used to simulate the effects of crowded sources on the bright central object is limited to 32nd magnitude, which originally was troublesome as LSST will reach depths of 27th magnitude at its deepest. Here we show that the original issue of requiring $\Delta m \leq 10$ -- a requirement imposed by \citet{2018MNRAS.481.2148W} -- can be sidestepped, at least for LSST, by consideration of the noise levels of a given observation. Simple noise considerations show that at the faint end of a survey it is only necessary to consider perturbers up to 5 magnitudes fainter than the central source, for a very conservative estimate of including sources up to 5\% the noise of the primary object in the PSF. Additionally, we derive a more accurate model to describe the perturbations of sources in the background-dominated case, crucial for ensuring the AUFs of the very crowded, faintest LSST objects are as well-modelled as possible.

\end{abstract}

%%%%%%%%%%%%%%%%%%%%%%%%%%%%%%%%%%%%%%%%%%%%%%%%%%%%%%%%%%%%%%%%%%%%%%%%%%%%%%%%
\section{Introduction}
While we could not empirically derive physically motivated AUFs for modelling the non-statistical effects influencing the recorded positions of sources in photometric catalogues, we can improve the model used to create the AUF component describing the offset of two measurements of the positions of a given source on the sky caused by the effects of blended sources in one or both catalogues. Here we consider two effects which will play a significant role in the application of a Bayesian cross-matching algorithm to the identification of sources across two potentially crowded photometric catalogues: first, the description of the algorithm used to derive the positions and fluxes of the central source, implicitly encoding in its derived quantities the effects of any blended objects unresolved by the pipeline of the given survey; and second, the effect that noise will play on the limits on the influence of vanishingly faint objects on these same quantities.

\section{Deriving a more accurate AUF in the sky background limited case}
Previously, the algorithm used by \cite{2018MNRAS.481.2148W} to create fake Monte Carlo PSFs from which the average quantities affecting the systematic offsets of the recorded positions of a given astrophysical object used a relatively simple and naive flux-weighted average scheme. The astrometric \textit{perturbation} of a central source, caused by fainter unresolved objects within a \citet{1880MNRAS..40..254R} criterion of 1.185 full-width at half maximums (FWHMs)\footnote{Although some surveys, due to non-Gaussian wings in the telescope PSF, may have slightly larger unresolveable regions; \textit{WISE}, for instance, cannot resolve sources within 1.3FWHM -- see \citet{Cutri:2012aa}, section 4.4c.} was simply given by $x_\mathrm{pertub} = (0 + \sum_i x_i f_i) / (1 + \sum_i f_i)$ where $f$ is the relative flux between the central object, defined as $f=1$, and the faint perturber. The photometric \textit{contamination} was quoted as being $\sum_i f_i$ -- i.e., it was assumed that if there was 10\% additional flux inside the system, either as two sources with 5\% relative flux or a single object a tenth the brightness, the resultant photometric catalogue was 1.1 times too bright.

This simple model should be true for the case of aperture photometry, where any extra flux \textit{is} simply added to the total brightness of the central object when counting the flux within the aperture. The first moment -- used to derive the position of the source -- will result in essentially a flux-weighted average position, taking into consideration edge cases where a bright object appears just inside the aperture radius. However, if a more complicated model is used, and sources are fit with a PSF to derive their flux and centroid position then this assumption may not hold. Indeed, the case considered in \citet{2018MNRAS.481.2148W} is that of \textit{WISE}, which has a robust PSF fitting routine for its derived positions and flux measurements.

\subsection{The log-likelihood PSF fitting method}
Taking the slightly simplifying assumption that the PSF in question is described by a Gaussian, we can state the fitting process as a minimisation problem, which can equally be thought of as a likelihood maximisation problem. This equation, using a similar notation to that given by \citet{2018MNRAS.476.4372}, equation 1, is

\begin{equation}
    \mathrm{log}\,\mathcal{L} = -\frac{1}{2}\times L \int\limits_{-\infty}^\infty\! \left[\phi(\mathbf{r}) + f\phi(\mathbf{r - d}) - (1 + \Delta f)\phi(\mathbf{r - \Delta d})\right]^2 \mathrm{d}^2r,
\label{eq:logL1}
\end{equation}
where we are minimising the square of the differences between two models, one with unity flux (up to a scaling factor of $L$) at the origin and another with relative flux $f$ at position vector $\mathbf{d}$, and a single model with brightening $\Delta f$ at perturbation vector $\mathbf{r - \Delta d}$. Here $\phi$ is the equation describing the circularly symmetric Gaussian PSF,

\begin{equation}
    \phi(\mathbf{r}) = \phi(\mathbf{r}, \sigma_\phi) = \frac{1}{2\pi \sigma_\phi}\mathrm{exp}\left(-\frac{1}{2} \frac{\mathbf{r}^2}{\sigma_\phi^2}\right).
\end{equation}

Expanding the brackets results in six integrand terms, each of which has a multiplicitive factor in front of the multiple of two $\phi$ terms, with relative offsets (e.g., the second term would be $A\phi(\mathbf{r})\phi(\mathbf{r - d})$). These terms, integrated over all space, are the convolution of the two Gaussians, offset by some vector $\mathbf{d}$, and thus each of the six integrals can be analytically computed as six convolutions. As each $\phi$ term has the same uncertainty $\sigma_\phi$, the resultant Gaussian has an uncertainty of $\sqrt{2}\sigma_\phi$, as Gaussian convolutions result in a Gaussian with a variance the quadrature sum of the two composite variances. We therefore, for notation's sake, define a new term

\begin{align}
\begin{split}
    \psi(\mathbf{r}) &= \psi(\mathbf{r}, \sigma_\psi) = \frac{1}{2\pi \sigma_\psi}\mathrm{exp}\left(-\frac{1}{2} \frac{\lvert\mathbf{r}\lvert^2}{\sigma_\psi^2}\right) \\&\equiv \phi(\mathbf{r}, \sqrt{2}\sigma_\psi) = \frac{1}{4\pi \sigma_\phi}\mathrm{exp}\left(-\frac{1}{4} \frac{\lvert\mathbf{r}\lvert^2}{\sigma_\phi^2}\right).
\end{split}
\end{align}
Only three of the resulting convolutions contain terms with $\Delta f$ or $\mathbf{\Delta d}$ in them -- for example, the first term $L\phi(\mathbf{r})$ multiplied by itself, after evaluating the convolution through the integral, becomes $-\frac{1}{2} L^2 \psi(\mathbf{0})$. The three terms which are multiplications of two different terms within the square brackets of equation \ref{eq:logL1} appear twice and thus cancel the factor of a half in the definition of the log-likelihood. Combining all terms, evaluating all convolutions, and dropping constant terms, we therefore have

\begin{align}
\begin{split}
    \mathrm{log}\,\mathcal{L} = L \bigg[&(1 + \Delta f)\psi(\mathbf{\Delta d}) + (1 + \Delta f)\sum\limits_if_i\psi(\mathbf{d}_i - \mathbf{\Delta d})\,-\\&\frac{1}{2}(1 + \Delta f)^2\psi(\mathbf{0})\Bigg].
\end{split}
\end{align}

For small perturbations by very faint perturbers, we can derive analytic expressions for $\Delta f$ and $\Delta x$ (and $\Delta y$, to view sky coordinates in a small enough area to consider the region as cartesian) by taking derivatives with respect to those same values and setting to zero. For $\Delta f$ we have
\begin{align}
\begin{split}
    \frac{\partial\mathrm{log}\,\mathcal{L}}{\partial \Delta f} = L &\bigg[\psi(\mathbf{\Delta d}) + \sum\limits_if_i\psi(\mathbf{d}_i + \mathbf{\Delta d}) - (1 + \Delta f)\psi(\mathbf{0})\bigg] = 0.
\end{split}
\end{align}
Defining $\psi{'}(\mathbf{x}) \equiv \psi(\mathbf{x})/\psi(\mathbf{0})$, $\Delta f$ can be solved for as
\begin{align}
\begin{split}
    &\psi{'}(\mathbf{\Delta d}) + \sum\limits_if_i\psi{'}(\mathbf{d}_i + \mathbf{\Delta d}) - (1 + \Delta f) = 0\\
    &\Delta f = \psi{'}(\mathbf{\Delta d}) - 1 + \sum\limits_if_i\psi{'}(\mathbf{d}_i + \mathbf{\Delta d}).
\label{eq:dfderiv}
\end{split}
\end{align}
In the limit that $\Delta d \ll 1$, $\psi{'}(\mathbf{\Delta d}) \to 1$ and $\psi{'}(\mathbf{d}_i + \mathbf{\Delta d}) \to \psi{'}(\mathbf{d}_i)$ and thus

\begin{equation}
    \Delta f \simeq \sum\limits_if_i\psi{'}(\mathbf{d}_i) = \sum\limits_if_i\mathrm{exp}\left(-\frac{1}{4}\frac{\lvert\mathbf{d}_i\lvert^2}{\sigma_\psi^2}\right),
\end{equation}
as quoted by \citet{2018MNRAS.476.4372}, equation 3. Similarly we can differentiate with respect to one or other cartesian coordinate,
\begin{align}
\begin{split}
    \frac{\partial\mathrm{log}\,\mathcal{L}}{\partial \Delta x} = L \bigg[&(1 + \Delta f)\frac{\Delta x}{2\sigma_\phi^2}\psi(\mathbf{\Delta d})\,-\\&(1 + \Delta f)\sum\limits_if_i\frac{x_i - \Delta x}{2\sigma_\phi^2}\psi(\mathbf{d}_i - \mathbf{\Delta d})\Bigg] = 0.
\end{split}
\end{align}
Rearranging and using the previous definition of $\psi{'}$ we get
\begin{align}
\begin{split}
    &(1 + \Delta f)\frac{\Delta x}{2\sigma_\phi^2}\psi{'}(\mathbf{\Delta d}) = (1 + \Delta f)\sum\limits_if_i\frac{x_i - \Delta x}{2\sigma_\phi^2}\psi{'}(\mathbf{d}_i - \mathbf{\Delta d})\\
    &\Delta x\,\psi{'}(\mathbf{\Delta d}) = \sum\limits_if_i(x_i - \Delta x)\psi{'}(\mathbf{d}_i - \mathbf{\Delta d}).
\label{eq:dxderiv}
\end{split}
\end{align}
Again assuming $\Delta d \ll 1$ and thus $\psi{'}(\mathbf{\Delta d}) \to 1$, $x_i - \Delta x \to x_i$, and $\psi{'}(\mathbf{d}_i - \mathbf{\Delta d}) \to \psi{'}(\mathbf{d}_i)$, we have
\begin{equation}
    \Delta x \simeq \sum\limits_if_i\Delta x_i\,\psi{'}(\mathbf{d}_i) = \sum\limits_if_ix_i\mathrm{exp}\left(-\frac{1}{4}\frac{\lvert\mathbf{d}_i\lvert^2}{\sigma_\psi^2}\right).
\end{equation}

While these approximations are valid for small $f$, they are not applicable to cases of approximately equal-brightness contaminants; we therefore require an approximation for the perturbation of these bright objects in the sky background-limited case.

STUFF

\section{Including the effects of noise in the simulating of perturbation AUF components}
While the improved model for fainter source perturbation modelling is important, a pragmatic consideration we must make when considering the cross-matching of LSST catalogues to external catalogues is that of dynamic range. LSST is expected to reach 5-sigma depths of 27th magnitude, which leads to a potential issue with Galactic source count simulations. Previously, \citet{2018MNRAS.481.2148W} considered the TRILEGAL \citep{Girardi2005} Galactic simulations for star counts; however, the dynamic range of these simulations -- at least, those publically available -- is 32nd magnitude. \citet{2018MNRAS.481.2148W} also recommended simulating faint perturbers up to 10 magnitudes fainter than the primary object, for source astrometric precision arguments. Sources sufficiently faint are not available through the TRILEGAL simulations, so another approach is needed. However, the previous PSF simulations do not take into account any noise in the images, which we now consider.

\subsection{Noise effects on faint object astrometric perturbations}
For these purposes we consider a simple system: a primary object, of flux $F_p$ (or scaling amplitude $A$, with $F_p$ the integral of the PSF times $A$), and a secondary object some distance $d$ offset with flux $F_s$ (or scaling amplitude $fA$, where $f$ is the ratio of secondary-to-primary fluxes). When calculating the perturbation component of the AUF, sources are randomly simulated within a given PSF down to some flux ratio and the perturbation position recorded in some way. However, this was previously done in a noiseless environment, with flux-weighted averages considered; the faint limit cutoff was therefore slightly arbitrary. ``Real'' PSF fitting, however, is done in an environment that contains noise -- both from the physical sky objects, and the background sky itself. Thus we can now consider the limit case where the perturbing object cannot be seen below the noise in the primary -- or composite -- object,
\begin{equation}
    F_s \geq B\sqrt{F_p + S + fF_pQ},
\end{equation}
where $B$ is a factor, less than one, which dictates how many sigma we need to consider a faint object to, $S$ is the sky flux or source counts within the PSF area in question, and $Q$ is the fraction of the flux from the secondary within the PSF area of the primary (typically computed as the integral of a PSF centered on the secondary over the primary PSF region). For our faint source cases $f \ll 1$ so we can make the assumption that all of the noise comes from the primary object or sky, and thus
\begin{equation}
    F_s \geq B\sqrt{F_p + S} = B\sigma_{F_p}.
\end{equation}
Considering $f \equiv F_s/F_p$ we can divide both sides by $F_p$ to obtain
\begin{equation}
    f \geq B\frac{\sigma_{F_p}}{F_p} = B\,\mathrm{SNR}_p^{-1}.
\end{equation}

As most photometric surveys are defined by their 5-sigma completeness limits, we need to focus on $\mathrm{SNR}_p \geq 5$. Also considering a swap from relative flux ratio to magnitude offset, $\Delta m = -2.5 \log_{10}(f)$, we can also select $B=0.05$ -- or sources at minimum a twentieth the noise of the primary -- as this gives $\Delta m = -2.5 \log_{10}(0.05/5) = 5$. Additionally, at $\mathrm{SNR}_p = 500$ we ``only'' need to consider sources down to $\Delta m = 10$, and thus this $B$ provides a reasonable range of faint perturbation flux limits as a function of primary flux.

\subsection{Noise effects on faint object photometric contamination}
While the above equation is sensible in its determination of the limit of a faint perturber on the astrometric quantities of a bright object, we must also verify the flux contamination derived is robust to these limits of flux ratio. Understandably, at high fluxes for the primary object, 


%%%%%%%%%%%%%%%%%%%%%%%%%%%%%%%%%%%%%%%%%%%%%%%%%%%%%%%%%%%%%%%%%%%%%%%%%%%%%%%%

\bibliographystyle{mnras}
\bibliography{../../../../Dropbox/PostDoc_Exeter/LaTeX/postdoc_papers_jr2}



% Don't change these lines
\bsp	% typesetting comment
\label{lastpage}
\end{document}

% End of mnras_template.tex